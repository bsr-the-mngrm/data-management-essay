\documentclass{article}
\usepackage{graphicx} % Required for inserting images

\title{data-dynamics-in-startups-and-SMEs}
\author{Sándor Róbert Bakos}
\date{November 2024}

\begin{document}

\maketitle

\section*{Introduction} \cite{openai2024chatgpt}

In today’s digital economy, data has emerged as a critical asset for businesses of all sizes \cite{mcafee2012bigdata}. Whether it is customer insights, operational metrics, or market trends, effectively managing data is essential for organizations to remain competitive \cite{redman2013credibility}. Data management encompasses the practices, tools, and frameworks used to collect, store, process, and analyze data to derive valuable information \cite{dama2017dmbook}. However, the challenges of managing data vary significantly depending on the size and nature of the organization \cite{davenport2014bigdata}.

Startups are often defined by their agility, limited resources, and rapid growth trajectories, which influence their approach to data management \cite{ries2011leanstartup}. On the other hand, middle-sized companies typically operate on more stable grounds with established workflows and resources but face complexities in scaling and maintaining their data systems \cite{european2015smedef}. These differences necessitate tailored data management strategies that align with each organization's unique operational structures and goals.

This essay aims to explore and compare the data management challenges faced by startups and middle-sized companies. By analyzing their strengths and limitations, it will highlight how their respective approaches to data management impact their operations. Additionally, it will examine how startups and middle-sized companies can overcome these challenges and adopt best practices to improve their data management strategies.



\bibliographystyle{plain}
\bibliography{reference}

\end{document}
